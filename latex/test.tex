\documentclass{article}
\usepackage[UTF8]{ctex}

%要运行该模板,LaTex需要安装CJK库以支持汉字.
%字体大小为12像素,文档类型为article
%如果你要写论文,就用report代替article
%所有LaTex文档开头必须使用这句话
%使用支持汉字的CJK包

%开始CJK环境,只有在这句话之后,你才能使用汉字
%另外,如果在Linux下,请将文件的编码格式设置成GBK
%否则会显示乱码

	%这是文章的标题
	\title{HPC Final Project}
	%这是文章的作者
	\author{12132431 钟昊辰}
	%这是文章的时间
	%如果没有这行将显示当前时间
	%如果不想显示时间则使用 \date{}
	\date{2022/6/9}
	%以上部分叫做"导言区",下面才开始写正文
	\begin{document}
		%先插入标题
		\maketitle     %主要的作用是用于生成标题的作用 content contain \title \author \date
		%再插入目录
		
		\tableofcontents   %主要的作用适用于生成目录的作用
		
		\clearpage
		
		\section{Problem B :陈述}
		考虑一维(1D)域中的热传导方程$\Omega:=(0,1)$.域的边界为$\Gamma={0,1}$.设$f$为单位体积的热源,$u$ 为温度,它是关于x和t的函数;$\rho$为密度,$c$为热容,$u_0$为初始温度,$\kappa$为传热系数,$n_x$为笛卡尔坐标系下的单位法向量.边界条件为$\Gamma_g$上规定的温度函数$g$和$\Gamma_h$上的热通量函数$h$,它们均为x和t的函数.边界$\Gamma$允许非重叠分解:$ \Gamma = \Gamma_g \cap \Gamma_h,\emptyset = \Gamma_g \cup \Gamma_h$.热传导方程可以表述如下:
		\\
		$
		\rho c \frac{\partial u}{\partial t}-\kappa\frac{\partial^2 u}{\partial x^2}=f
		$ \quad on $\Omega $
		
		
		
		\clearpage
		
		\section{代码构成}
		
		LaTex 源文件格式为普通的ASCII文件,
		你可以使用任何文本编辑器来创建.
		LaTex源文件不仅包括你要排版的文本, 还包括LaTex
		所能识别的,如何排版这些文本的命令.
		
		\clearpage
		
		\section{代码测试}
		\subsection{方法稳定性}
		小节2.1
		\subsection{误差分析}
		小节2.2
		%在结论部分我们使用仿宋体
		LaTeX, 我看行!
		
		\clearpage
		
		\section{并行化测试}
		
		%在结论部分我们使用仿宋体
		LaTeX, 我看行!
		
\end{document}
